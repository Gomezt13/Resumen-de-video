\documentclass[12pt, a4paper]{article}
\usepackage[spanish]{babel}
\usepackage[utf8]{inputenc}
\usepackage{graphics}
\usepackage[margin=2.54cm]{geometry}


\begin{document}

    \section{Resumen del primer video}
    Naturaleza contextual de la motivación: La motivación varía en función del contexto, la historia personal, el tiempo y los objetivos específicos. No es un concepto fijo, sino que depende de varios factores.

    Cambios impactantes en el entorno: La modificación del entorno y de la organización puede afectar significativamente a la motivación. Cambios sencillos, como que las enfermeras lleven chalecos durante la administración de la medicación, redujeron los errores en un 47%, lo que demuestra el poder de las alteraciones del entorno.

    Motivadores: Dinero y reconocimiento social: El dinero y el reconocimiento social son motivadores inagotables. Las recompensas económicas y el reconocimiento social impulsan sistemáticamente la motivación en los entornos profesionales.

    Toxicidad y comportamiento contextual: La toxicidad no es inherente a los individuos, sino a los comportamientos, contextos y entornos de trabajo. Un cambio en las circunstancias puede hacer que una persona pase de ser productiva a tóxica, lo que pone de relieve el papel del contexto en el comportamiento.

    Diferenciar estado de ánimo y motivación: Distinguir entre estado de ánimo y motivación es crucial. Mientras que el estado de ánimo puede verse influido por la actividad física y los gestos, la motivación requiere planificación y un sentido de propósito, lo que indica que motivar a alguien va más allá de animarle.

    Estos puntos ponen de relieve la naturaleza matizada y contextual de la motivación y subrayan la necesidad de comprender los factores individuales y las influencias del entorno.
    
    \section{ Resumen del segundo video}
    La lección 7 se centra en los factores humanos y la motivación en las relaciones interpersonales.

    Los directivos eficaces suelen ser líderes eficaces, pero la gestión implica algo más que dirigir; abarca influir en los individuos para que alcancen los objetivos de la organización en un entorno de grupo.

    La distinción entre un grupo y un equipo radica en los objetivos comunes: un grupo puede no compartir objetivos comunes, mientras que un equipo sí, trabajando colectivamente hacia un resultado específico.

    En la gestión, los individuos son miembros integrales del sistema social, no sólo factores productivos; desempeñan diversas funciones en la sociedad y en diversos entornos.

    La motivación varía de un individuo a otro en función de sus objetivos, deseos y ambiciones personales, lo que da lugar a diferentes enfoques del trabajo y del compromiso dentro de una organización.
\end{document}